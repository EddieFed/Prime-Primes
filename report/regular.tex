\documentclass[11pt]{article}
\usepackage[paperwidth=10.2in,paperheight=11.7in]{geometry}


\usepackage{framed}

\usepackage{amsmath, amssymb}
\usepackage{listings}
\usepackage{color}
\usepackage{graphicx}

\setlength{\parindent}{0pt}
\setlength{\parskip}{12pt}
\setlength{\voffset}{-30pt}
\setlength{\hoffset}{-40pt}
\setlength{\textwidth}{600pt}
\setlength{\textheight}{720pt}
\setlength{\marginparwidth}{5pt}

\definecolor{codegreen}{rgb}{0,0.6,0}
\definecolor{codegray}{rgb}{0.5,0.5,0.5}
\definecolor{codepurple}{rgb}{0.58,0,0.82}
\definecolor{backcolour}{rgb}{0.95,0.95,0.92}
\lstdefinestyle{EddiePythonStyle}{
    backgroundcolor = \color{backcolour},
    commentstyle = \color{codegreen},
    keywordstyle = \color{magenta},
    numberstyle = \tiny\color{codegray},
    stringstyle = \color{codepurple},
    basicstyle = \ttfamily\footnotesize,
    breakatwhitespace = false,         
    breaklines = true,                 
    captionpos = b,                    
    keepspaces = true,                 
    numbers = left,                    
    numbersep = 6pt,                  
    showspaces = false,                
    showstringspaces = false,
    showtabs = false,                  
    tabsize = 2,
    language = Python,
    frame = single,
    title = {Python code to generate Bernoulli numbers as well as using them to check if a prime is regular}
}

\newcommand{\genlegendre}[4]{%
    \genfrac{(}{)}{}{#1}{#3}{#4}%
    \if\relax\detokenize{#2}\relax\else_{\!#2}\fi
}
\newcommand{\legendre}[3][]{\genlegendre{}{#1}{#2}{#3}}
\newcommand{\dlegendre}[3][]{\genlegendre{0}{#1}{#2}{#3}}
\newcommand{\tlegendre}[3][]{\genlegendre{1}{#1}{#2}{#3}}

\begin{document}
    \lstset{style=EddiePythonStyle}
    \begin{lstlisting}
#!/usr/bin/env python
# -*- coding: utf-8 -*-

from fractions import Fraction as Fr

"""
bernoulli_num(n) generates the first n bernoulli numbers and stiores only the even bernoulli number
numerators (B(k) where k = 2n) in a file names bernoulli.txt. This script is a slightly modified
version of the code found at https://rosettacode.org/wiki/Bernoulli_numbers#Python. Personally, I
didn't have any use for the denominator of the bernoulli numbers so I simply ignore them. This is
much more optimal for space, however, the algorithm is still stunted in terms of speed since it's
practically exponential, taking nearly 9 hours to generate B(0) -> B(10,000) on an intel i7-6700k
"""

n = 1000    # Number of bernoulli numbers to generate.
            # The numerators wil be written to bernoulli.txt

def bernoulli_generator():
    A, m = [], 0
    while True:
        A.append(Fr(1, m+1))
        for j in range(m, 0, -1):
            A[j-1] = j*(A[j-1] - A[j])
        yield A[0] # (which is Bm)
        m += 1

def bernoulli_num(n: int, to_file = True) -> list[int]:
    bn = [ix for ix in zip(range(n), bernoulli_generator())]
    bn = [(i, b) for i,b in bn if b]

    if to_file:
        with open("bernoulli.txt", "w+") as file:
            index = 0
            for i, b in bn:
                # print('B(%2i) = %*i/%i' % (i, width, b.numerator, b.denominator))
                if (index > 1):
                    file.write("%s\n" % str(b.numerator))
                index += 1
    else:
        _bn = []
        for i, b in bn:
            _bn.append(b.numerator)
        return _bn
    \end{lstlisting}

    \newpage
    \lstset{style=EddiePythonStyle}
    \begin{lstlisting}
#!/usr/bin/env python
# -*- coding: utf-8 -*-

import os

"""
This is a standalone script not meant to be called independently. This script reads
in all primes generated from sieve() and all bernoulli number numerators from
bernoulli_num(). It will spit out a new text file named regular.txt which contains
only regular prime numbers.

Regular Primes:
    Def => Prime numbers p which do not divide the numerator of any bernoulli number B(k)
    for all k where k = 2n and k <= p-3.

    For more info see: https://oeis.org/A007703
"""

# If neither file exists, abort!
if (not os.path.exists("bernoulli.txt") or not os.path.exists("primes.txt")):
    print("Please ensure that bernoulli.txt and primes.txt are in the root directory!")
    exit()

bernoulli_n = []
with open("bernoulli.txt") as file:
    for line in file:
        bernoulli_n.append(int(line))

primes = []
with open("primes.txt") as file2:
    for line2 in file2:
        primes.append(int(line2))

# Determine the largest prime which we are allowed to check!
p_3 = len(bernoulli_n)*2 + 3

regular_primes = []
for p in primes:
    if (p <= p_3):
        if (p-3 >= 0):  # This will exlude 2 since B(2-3) is not a valid bernoulli number
            is_reg = True
            # print(f"Checking if {p} is regular")

            # Check the divisibility of all even bernoulli numbers less than p-3
            index = 2 
            for b in bernoulli_n:

                # Only check B(k) for 2, 4, 6, ..., p-3
                if (index > p-3):
                    break

                # If p divides B(k), than it is not a regular prime
                if (b % p == 0):
                    is_reg = False
                    break

                # This is just to keep track of p < p-3
                index += 2
        else:
            is_reg = False

        if (is_reg):
            # print(f"{p} is regular")
            regular_primes.append(p)

    else:
        break

# Dump it all to a file!
with open("reg.txt", "w+") as file:
    for reg in regular_primes:
            file.write("%s\n" % str(reg))
    \end{lstlisting}

\end{document}

